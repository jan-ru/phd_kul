% Options for packages loaded elsewhere
\PassOptionsToPackage{unicode}{hyperref}
\PassOptionsToPackage{hyphens}{url}
\PassOptionsToPackage{dvipsnames,svgnames,x11names}{xcolor}
%
\documentclass[
  letterpaper,
  DIV=11,
  numbers=noendperiod]{scrartcl}

\usepackage{amsmath,amssymb}
\usepackage{iftex}
\ifPDFTeX
  \usepackage[T1]{fontenc}
  \usepackage[utf8]{inputenc}
  \usepackage{textcomp} % provide euro and other symbols
\else % if luatex or xetex
  \usepackage{unicode-math}
  \defaultfontfeatures{Scale=MatchLowercase}
  \defaultfontfeatures[\rmfamily]{Ligatures=TeX,Scale=1}
\fi
\usepackage{lmodern}
\ifPDFTeX\else  
    % xetex/luatex font selection
\fi
% Use upquote if available, for straight quotes in verbatim environments
\IfFileExists{upquote.sty}{\usepackage{upquote}}{}
\IfFileExists{microtype.sty}{% use microtype if available
  \usepackage[]{microtype}
  \UseMicrotypeSet[protrusion]{basicmath} % disable protrusion for tt fonts
}{}
\makeatletter
\@ifundefined{KOMAClassName}{% if non-KOMA class
  \IfFileExists{parskip.sty}{%
    \usepackage{parskip}
  }{% else
    \setlength{\parindent}{0pt}
    \setlength{\parskip}{6pt plus 2pt minus 1pt}}
}{% if KOMA class
  \KOMAoptions{parskip=half}}
\makeatother
\usepackage{xcolor}
\setlength{\emergencystretch}{3em} % prevent overfull lines
\setcounter{secnumdepth}{-\maxdimen} % remove section numbering
% Make \paragraph and \subparagraph free-standing
\ifx\paragraph\undefined\else
  \let\oldparagraph\paragraph
  \renewcommand{\paragraph}[1]{\oldparagraph{#1}\mbox{}}
\fi
\ifx\subparagraph\undefined\else
  \let\oldsubparagraph\subparagraph
  \renewcommand{\subparagraph}[1]{\oldsubparagraph{#1}\mbox{}}
\fi


\providecommand{\tightlist}{%
  \setlength{\itemsep}{0pt}\setlength{\parskip}{0pt}}\usepackage{longtable,booktabs,array}
\usepackage{calc} % for calculating minipage widths
% Correct order of tables after \paragraph or \subparagraph
\usepackage{etoolbox}
\makeatletter
\patchcmd\longtable{\par}{\if@noskipsec\mbox{}\fi\par}{}{}
\makeatother
% Allow footnotes in longtable head/foot
\IfFileExists{footnotehyper.sty}{\usepackage{footnotehyper}}{\usepackage{footnote}}
\makesavenoteenv{longtable}
\usepackage{graphicx}
\makeatletter
\def\maxwidth{\ifdim\Gin@nat@width>\linewidth\linewidth\else\Gin@nat@width\fi}
\def\maxheight{\ifdim\Gin@nat@height>\textheight\textheight\else\Gin@nat@height\fi}
\makeatother
% Scale images if necessary, so that they will not overflow the page
% margins by default, and it is still possible to overwrite the defaults
% using explicit options in \includegraphics[width, height, ...]{}
\setkeys{Gin}{width=\maxwidth,height=\maxheight,keepaspectratio}
% Set default figure placement to htbp
\makeatletter
\def\fps@figure{htbp}
\makeatother

\usepackage[ style        = ext-numeric, backend      = biber, defernumbers = true, ]{biblatex} \usepackage{totcount}
\KOMAoption{captions}{tableheading}
\makeatletter
\makeatother
\makeatletter
\makeatother
\makeatletter
\@ifpackageloaded{caption}{}{\usepackage{caption}}
\AtBeginDocument{%
\ifdefined\contentsname
  \renewcommand*\contentsname{Table of contents}
\else
  \newcommand\contentsname{Table of contents}
\fi
\ifdefined\listfigurename
  \renewcommand*\listfigurename{List of Figures}
\else
  \newcommand\listfigurename{List of Figures}
\fi
\ifdefined\listtablename
  \renewcommand*\listtablename{List of Tables}
\else
  \newcommand\listtablename{List of Tables}
\fi
\ifdefined\figurename
  \renewcommand*\figurename{Figure}
\else
  \newcommand\figurename{Figure}
\fi
\ifdefined\tablename
  \renewcommand*\tablename{Table}
\else
  \newcommand\tablename{Table}
\fi
}
\@ifpackageloaded{float}{}{\usepackage{float}}
\floatstyle{ruled}
\@ifundefined{c@chapter}{\newfloat{codelisting}{h}{lop}}{\newfloat{codelisting}{h}{lop}[chapter]}
\floatname{codelisting}{Listing}
\newcommand*\listoflistings{\listof{codelisting}{List of Listings}}
\makeatother
\makeatletter
\@ifpackageloaded{caption}{}{\usepackage{caption}}
\@ifpackageloaded{subcaption}{}{\usepackage{subcaption}}
\makeatother
\makeatletter
\@ifpackageloaded{tcolorbox}{}{\usepackage[skins,breakable]{tcolorbox}}
\makeatother
\makeatletter
\@ifundefined{shadecolor}{\definecolor{shadecolor}{rgb}{.97, .97, .97}}
\makeatother
\makeatletter
\makeatother
\makeatletter
\makeatother
\ifLuaTeX
  \usepackage{selnolig}  % disable illegal ligatures
\fi
\IfFileExists{bookmark.sty}{\usepackage{bookmark}}{\usepackage{hyperref}}
\IfFileExists{xurl.sty}{\usepackage{xurl}}{} % add URL line breaks if available
\urlstyle{same} % disable monospaced font for URLs
\hypersetup{
  pdftitle={Software Support},
  colorlinks=true,
  linkcolor={blue},
  filecolor={Maroon},
  citecolor={Blue},
  urlcolor={Blue},
  pdfcreator={LaTeX via pandoc}}

\title{Software Support}
\author{}
\date{30/03/2023}

\begin{document}
\maketitle
\ifdefined\Shaded\renewenvironment{Shaded}{\begin{tcolorbox}[borderline west={3pt}{0pt}{shadecolor}, enhanced, boxrule=0pt, interior hidden, sharp corners, frame hidden, breakable]}{\end{tcolorbox}}\fi

\hypertarget{tools}{%
\subsection{{[}27-05-2023{]} Tools}\label{tools}}

(https://pypi.org/project/refextract/)

(https://docs.readme.com/rdmd/docs/getting-started)

ORCID: reference use of the latex package ``orcidlink''.

Idea for backlog: obtain keywords via orcid api (python) and build
wordcloud.

\hypertarget{day08-include-academicons}{%
\subsection{{[}26-05-2024 / day08{]} Include
Academicons}\label{day08-include-academicons}}

\href{https://schochastics.quarto.pub/academicons-quarto-extension/}{quarto
extention}

Potentially add google scholar link to author index

Possibly also use
\href{https://www.w3schools.com/icons/fontawesome5_icons_computers.asp}{fontawesome
computer icons}

Look into
\href{https://shafayetshafee.github.io/interactive-sql/example.html}{interactive-sql}
which may help to clarify relationships between accounts and
``verdichtingen''. Note that interactive-sql is a wrapper aroung
\href{https://sqlime.org/about.html}{sqlime}.

\hypertarget{document-attributes}{%
\subsection{Document attributes}\label{document-attributes}}

Current thinking: use latex package

\begin{itemize}
\item
  \regtotcounter{table}
\item
  \regtotcounter{figure}
\end{itemize}

In case of output to html, and not to loose functionality of certain
latex packages (such as orcid):

\begin{itemize}
\tightlist
\item
  Document is processed and variables obtain a number.
\item
  Intermediary file is saved.
\item
  Variables are read
  \href{https://tex.stackexchange.com/questions/321346/how-to-read-a-variable-from-a-file-in-latex}{ref.
  texexchange}
\item
  html snippet creation \href{https://texfaq.org/FAQ-LaTeX2HTML}{html
  snippet}
\item
  html snippet is included in header, footer, or body of file
\end{itemize}

\hypertarget{bibliometrics}{%
\subsection{Bibliometrics}\label{bibliometrics}}

To find the number of publications they both worked on I may want to
look into \href{https://www.bibliometrix.org/home/}{bibliometrix}. This
is in R, there maybe similar tools for Python. Pybibliometrics \href{}{}
looks to be an interface to Elsevier Scopus only.

\hypertarget{references}{%
\subsection{References}\label{references}}

https://www.tug.org/texlive//Contents/live/texmf-dist/doc/latex/orcidlink/orcidlink.pdf

\hypertarget{number-of-bib-entries}{%
\subsection{Number of bib entries}\label{number-of-bib-entries}}

\newcounter{totalbibentries}
\newcommand*{\listcounted}{}

\makeatletter
\AtDataInput{%
  \xifinlist{\abx@field@entrykey}
    {}
    {\stepcounter{totalbibentries}%
     \listxadd{\abx@field@entrykey}}%
}
\makeatother

\begin{filecontents}{\jobname.bib}
@article{A,
  author          = {A Author},
  keywords        = {A1},
}
@article{B,
  author          = {B Author},
  keywords        = {A2},
}
@article{C,
  author          = {C Author},
  keywords        = {A2,A3},
}
@article{D,
  author          = {D Author},
  keywords        = {A1,A3},
}
\end{filecontents}
\addbibresource{\jobname.bib}

\begin{document}
There are in total 
    \thetotalbibentries\
publications. 

\newcommand\crcbib[1]{%
  \begin{refsection}
    \section{Project #1}
    \nocite{*}
    \printbibliography[%
      resetnumbers = true,
      heading = none,
      keyword = {#1},
    ]
  \end{refsection}
} 
%
  \begin{refsection}
    \section{Project A1}
    \nocite{*}
    \printbibliography[%
      resetnumbers = true,
      heading = none,
      keyword = {A1},
    ]
  \end{refsection}

%
  \begin{refsection}
    \section{Project A2}
    \nocite{*}
    \printbibliography[%
      resetnumbers = true,
      heading = none,
      keyword = {A2},
    ]
  \end{refsection}

%
  \begin{refsection}
    \section{Project A3}
    \nocite{*}
    \printbibliography[%
      resetnumbers = true,
      heading = none,
      keyword = {A3},
    ]
  \end{refsection}

\end{document}



\end{document}
